%\documentclass[12pt,a4paper]{report}
\documentclass[12pt,a4paper]{book}
\setcounter{secnumdepth}{4}
\setcounter{tocdepth}{2}
% Пакети
%\usepackage{a4wide} %за по-широк текст
\usepackage{titlesec}
\titleformat{\chapter}{\LARGE\bfseries}{\thechapter}{1em}{}

\usepackage{fullpage}
\usepackage{amsmath}
\usepackage{amsfonts}
\usepackage{amssymb}
\usepackage{amsthm}
\usepackage[bulgarian]{babel}
\usepackage[utf8]{inputenc}
\usepackage[T1]{fontenc} %за кавичките
\usepackage[pdftex]{graphicx} %за картинки
\usepackage{hyperref}
\usepackage{textcomp}

%\usepackage{graphicx}
\usepackage{caption}
\usepackage{subcaption}
%\usepackage{multirow}

\hypersetup{colorlinks,
linkcolor=black,
urlcolor=black,
citecolor=black,
filecolor=black}
\begin{document}
\input{titlePage.tex}
%\begin{abstract}
%content...
%\end{abstract}
\tableofcontents
\chapter{Увод}
\section{Стандартен модел на елементарните частици}
\par Стандартният модел на елементарните частици (СМ) \cite{PDG,Pich} описва взаимодействието на съставните части на материята. В рамките на СМ, материята е изградена от $12$ фундаментални частици със спин $\frac{1}{2}$ - фермиони, които могат да бъдат обединени в три поколения (виж. фигура \ref{fig:standModel}). Фундаменталните фермиони са лептоните и кварките. Съществуват три лептона с електрически заряд ($-1$) - електрон ($e$), мюон($\mu$) и тау лептон ($\tau)$ и три електрически неутрални лептона - електронно ($\nu_e$), мюонно ($\nu_{\mu}$) и тау ($\nu_{\tau}$) неутрино. Кварките се разделят на два типа - горни и долни. Към горния тип кварки спадат горен ($u$ - up), чаровен ($c$ - charm) и върховен ($t$ - top) кварк, които носят електрически заряд ($+\frac{2}{3}$). Долният тип кварки имат електрически заряд ($-\frac{1}{2}$) и към тях спадат долен ($d$- down), странен ($s$- strange) и красив ($b$ - beauty) кварк.  Всяко едно поколение включва по един долен и един горен кварк и по един електрически зареден и един електрически неутрален лептон. Кварките участват в силните, слабите и електромагнитните взаимодействия. Силното взаимодействие се описва от Квантовата хромодинамика (КХД) \cite{gellmann,Politzer}. КХД е квантово полева теория, която използва неабелева калибровъчна група $SU(3)_C$, където индексът $C$ означава, че преобразованията на групата $SU(3)$ действат върху цветните състояния на кварките. В рамките на КХД кварките имат по три цвята, като взаимодействията между тях са пренасят от осем безмасови калибровъчни бозона - глуони. В следствие на неабелевия характер на калибровъчната група, глуоните също притежават цветен заряд и могат да взаимодействат помежду си. По тази причина константата на силното взаимодействие е по-малка при големи предадени импулси и е по-голяма при малки предадени импулси. Това е причината, поради която кварките не се срещат свободно в природата (т.н. кварков затвор), но образуват безцветни композитни състояния - мезони (състояние кварк-антикварк) и бариони (състояние от три кварка). Мезоните и барионите носят общото название адрони. В типичните протон-протонни взаимодействия при високи енергии, кварките получават достатъчно висок импулс. При опит за ,,раздалечаване`` на кварките от един адрон, енергията на връзката между тях е достатъчна за раждане на нова(нови) двойка(двойки) кварк-антикварк. Характерно за силните взаимодействия е така наречената ,,асимптотична свобода``, което означава, че при много малки разстояния, интензитетът на взаимодействия между кварките и глуоните е много малък, което е причина за възникване на ново състояние на материята - т.н. кварк-глуонна плазма.

\begin{figure}
\centering
\includegraphics[width=.7\textwidth]{/home/rumi/Desktop/teku6ti_dokumenti/atestatia/forPhD/disertations/diMuReso/standModel.jpg}
\caption{Стандартен модел на елементарните частици.}
\label{fig:standModel}
\end{figure}

\par Заредените лептони участват в електромагнитните и слабите взаимодействия, а неутрината - в слабите. Оказва се, че слабите взаимодействия не могат да бъдат разглеждани независимо от електромагнитните, което налага използването на модел, обединяващ двата типа взаимодействия. Електрослабите взаимодействия се описват от модела на Глешоу-Вайнберг-Салам \cite{Glashow,Weinberg,Salam}. Моделът използва калибровъчна теория с група на симетрия $SU(2)_L\times U(1)_Y$, с калибровъчни бозони $W_{\mu}^n,\, n = 1, 2, 3$ за $SU(2)$  и $B_{\mu}$ за $U(1)$ и константи на взаимодействие $g$ и $g'$.
Под действието на групата $SU(2)_L$ левите фермионни полета от $i-$тото поколение $\psi_i = \binom{\nu_i}{l_i^{-}}$ и $\binom{u_i}{d'_i}$ се преобразуват като дублети, където с $d'_i\equiv \sum_j V_{ij}d_j$ е отразено смесването между долните кварки от различните поколения, а $V$ е матрицата на Кабибо-Кобаяши-Маскава, която параметризира смесването на кварките. Под действието на групата $SU(2)_L$ десните фермионни полета се преобразуват като синглети.
\par Нарушаването на симетрията на електрослабата група и запазването на инвариантността на лагранжиана се постига посредством т.н. Механизъм на Брут-Енглерт-Хигс (Brout–Englert–Higgs mechanism) \cite{higgsOr,englert} чрез добавяне на скаларно комплексно поле $\phi$  с потенциал: 
\begin{equation}
V(\phi) = \mu^2 \phi^{\dagger}\phi + \frac{\lambda^2}{2} (\phi^{\dagger}\phi)^2
\end{equation}
\par При отрицателни стойности на $\mu^2$, за полето $\phi$ съществуват безкраен брой от изродени състояния с минимална енергия $v/\sqrt{2}$, където $v = 246.22$ GeV. Изборът на едно конкретно състояние нарушава спонтанно симетрията на електрослабата група до групата на Квантовата електродинамика (OED):
\begin{equation}
SU(2)_L\times U(1)_Y\,\,\, \underrightarrow{SSB}\,\,\, U(1)_{QED}
\end{equation}
Броят на генераторите с нарушена симетрия е три и според теоремата на Голдстоун това води до появата на три безмасови бозона със спин $0$. Скаларното комплексно поле $\phi$ може да бъде параметризирано с четири реални полета $\theta^i(x)$ и $H(x)$. Поради локалната инвариантност на лагранжиана относно групата $SU(2)_L$, той не зависи от трите полета $\theta^i(x)$. Тези полета се асоциират с трите безмасови Голдстоунови бозона. Взаимодействията на скаларното поле с калибровъчните бозони $W$ и $Z$  водят до ефективно появяване на надлъжна компонента, свързана с полетата  $\theta^i(x)$ и масов член за тях. Четвъртата компонента на скаларното поле е неутралната скаларна масивна частица $H$, която се нарича Хигс бозон. Лептоните и кварките също получават маса благодарение на Юкавското взаимодействие на фермионите с полето на Брут-Енглерт-Хигс.
\par След нарушаване на симетрията лагранжиана за фермионните полета може да се запише по следния начин \cite{PDG}:

\begin{align}
\label{eq:lagr_fermions}
\mathcal{L}_F &= \sum\limits_{i} \overline{\psi}_i \left(i \gamma^{\mu}\partial_{\mu} - m_i - \frac{g m_i H}{2 M_W}\right)\psi_i \nonumber\\
& -\frac{g}{2\sqrt{2}}\sum\limits_{i}\overline{\Psi}_i\gamma^{\mu}(1-\gamma^5)(T^{+}W_{\mu}^{+} + T^{-}W_{\mu}^{-})\Psi_i\\ \notag
& -e\sum\limits_{i}q_i\overline{\psi}_i\gamma^{\mu}\psi_i A_{\mu}\\ \notag
& -\frac{g}{2\cos \theta_W}\sum\limits_{i}\overline{\psi}_i\gamma^{\mu}(d^i_V - d^i_A \gamma^5)\psi_i Z_{\mu} \notag
\end{align}

където $\theta_W\equiv \tan^{-1}(g'/g)$ е ъгълът на смесване между слабото и електромагнитното взаимодействие; $e = g\sin \theta_W$ е електрическият заряд на позитрона; $A\equiv B\cos\theta_W + W^3\sin\theta_W$ е фотонното поле $(\gamma)$, а $W^{\pm}\equiv (W^1\mp i W^2)/\sqrt{2}$ и $Z\equiv -B\sin \theta_W + W^3\cos\theta_W$ са съответно заредените и неутралното бозонни полета.
\par В първия член на лагранжиана (\ref{eq:lagr_fermions}) $gm_i/2M_W$ отчита Юкавското взаимодействие на фермионите с полето на Брут-Енглерт-Хигс, $H$. 
%При евентуално съществуване на дясно неутрино този член ще включва и масата на Дираково неутрино. 
Необходимо е да се отбележи, че СМ разглежда неутрината като безмасови частици. 
В дървесно приближение масите на бозоните са както следва:
\begin{align}
& M_H = \lambda v ,\\
& M_W = \frac{1}{2} g v = \frac{e v}{2 \sin \theta_W},\\
& M_Z = \frac{1}{2}\sqrt{g^2+{g'}^2}v = \frac{e v}{2 \sin \theta_W \cos\theta_W} = \frac{M_W}{\cos \theta_W}\\
& M_{\gamma} = 0.
\end{align}
\par Вторият член от лагранжиана (\ref{eq:lagr_fermions}) описва слабите взаимодействия със заредени токове, като $T^{\pm} = 1/2(\tau_1\pm \tau_2)$ са съответно повишаващия и понижаващия изоспинов оператор, а $\tau_i$ са матрици на Паули.

%В частния случай, взаимодействието между електрон и неутрино ще се описват с лагранжиан:
%\begin{equation}
%-\frac{e}{2\sqrt{2}\sin \theta_W} \left[W_{\mu}^{-}\overline{e}\gamma^{\mu}(1-\gamma^5)\nu + W_{\mu}^{+}\overline{\nu}\gamma^{\mu}(1-\gamma^5)e\right]
%\end{equation}
\par Третият член от лагранжиана (\ref{eq:lagr_fermions}) описва електромагнитните взаимодействия, а четвъртият член описва взаимодействията с неутрални токове. С $g_V^i$ и $g_A^i$ са отразени съответно големината на векторното и аксиал-векторното взаимодействие, като:
\begin{align}
& g_V^i = T_3(i)-2q_i \sin^2\theta_W,\\
& g_A^i = T_3(i),
\end{align} 
където $T_3(i)=\tau_3/2$ е слабия изоспин на $i-$я фермион ($+1/2$ за $u_i$ и $\nu_i$; $-1/2$ за $d_i$ и $e_i$), $q_i$ е електрическият заряд на $\psi_i$.
\par През $2012$ г. експериментите CMS и ATLAS докладваха за откритие на нова частица с маса $125$ GeV/c$^2$, за която се счита, че е дълго търсеният Хигс бозон \cite{HiggsBoson, HiggsBoson1}. През $2013$ г. Нобеловата награда по физика бе присъдена на Франсоа Енглерт (François Englert) и Питър Хигс (Peter W. Higgs) за ,,теоретичното откритие на механизма, който допринася за нашето разбиране за появата на масата на субатомните частици, потвърден чрез откриването на предсказаната фундаментална частица от експериментите ATLAS и CMS в ЦЕРН на ускорителя LHC.``

\par Независимо, че СМ е най-успешната и експериментално потвърдена теория в областта на физиката на елементарните частици, все пак той не може да предложи адекватно решение на някои важни физични проблеми.
\begin{itemize}
\item Гравитация - Стандартния модел не включва в себе си описание на гравитационното взаимодействие;
\item Тъмна материя и тъмна енергия - От космологични оценки може да се направи извода, че СМ обяснява едва около $4\%$ от материята и енергията във Вселената;
\item Маси на неутриното - Съгласно СМ неутриното няма маса. Въпреки това последните неутринни експерименти ясно показват наличие на осцилации между различните аромати на неутриното, което е индикация за това, че неутрината имат маса;
\item Асиметрия между материя и антиматерия - На всяка частица от СМ съответства античастица, която има същата маса и същите свойства, но е с противоположни адитивни квантови числа. Взаимодействията на античастиците са със същия интензитет както и на съответните частици. Въпреки това нашата Вселена е изградена предимно от материя, като СМ не може да предложи достатъчно адекватно обяснение на този феномен.
%\item Свойства на Хигс бозона - Независимо от това, че съществуването на Хигс бозона беше потвърдено експериментално от колаборациите CMS и ATLAS, все още предстоят изследвания, касаещи свойствата на наблюдаваната частица, по конкретно на нейния спин и дали тя наистина съответства на предсказанията на СМ;
\item Проблема с йерархията на масите на фермионите - Характерните енергетични мащаби на различните фундаментални взаимодействия, а така също и на взаимодействията, описани от различни теории извън рамките на СМ, се различават на много порядъци. Частиците в СМ получават маса чрез механизма на спонтанно нарушаване на симетрията. Нарушаването на електрослабата симетрия става при енергии от порядъка на $\sim 10^2$ GeV, което определя и големината на масата на калибровъчните бозони. В сравнение с тази енергия, енергията при която се предполага обединението на силното с електрослабото взаимодействие е от порядъка на $\sim 10^{15}$ GeV, а обединението с гравитационното се предполага да се осъществява при мащаби от порядъка на  $\sim 10^{19}$ GeV. Причината за тази йерархия и за съществуването на т.н. ,,пустинни`` участъци между различните енергетични мащаби все още няма адекватно обяснение от СМ. Освен това при теоретичното определяне на масата на Хигс бозона, трябва да се отчетат и квантовите поправки, които се дължат на самодействието на Хигс бозона и на взаимодействието му с фермионите и калибровъчните бозони. Ако предположим, че СМ е валиден до някаква скала $\Lambda_{UV}$, всички радиационни поправки $\delta m_H$ са пропорционални на енергетичния мащаб $\Lambda_{UV}$, т.е. поправките към масата на Хигс бозона са много по-големи от неговата маса, определена в дървесно приближение.
\item CP проблема на силното взаимодействие - Независимо, че е възможно към лагранжиана на КХД да се допише инвариантен член, при който силното взаимодействие нарушава CP симетрията, до момента няма експериментални доказателства за подобен тип нарушение.
\end{itemize}
\section{Теории извън Стандартния модел}
\par Редица модели извън СМ се опитват да дадат отговор на посочените проблеми, като разширят или включат в себе си СМ като ниско енергетична граница. Някои от теориите, като т.н. Допълнен стандартен модел (Sequential standart model - SSM) \cite{SSM} предполагат допълването на СМ с нови по-тежки векторни бозони, като се запазват същите константите на взаимодействие. Допълнителните бозони $V^{\pm}, V^0$ могат да бъдат наблюдавани не само по типичните лептонни канали на разпад $V^{\pm}\rightarrow l^{\pm},\nu$ или $V^0\rightarrow l^{+},l^{-}$, а така също и по каналите  $V\rightarrow l^{\pm},\nu, jj$ и  $V\rightarrow l^{+}l^{-}jj$, които включват събития от типа $V^{\pm}\rightarrow W^{\pm}Z$ или $V^0\rightarrow WW$. При търсенето на тежки векторни бозони, чисто лептонните канали на разпад са предпочитани основно заради ясната крайна конфигурация и по-лека реконструкция на събитията.
\par Обединението на силното с електрослабото взаимодействие е възможно в рамките на {\bf Теории за великото  обединение (Grand Unified Theories -GUT's)} \cite{GUT} чрез разширяване на калибровъчната група на симетрия. Обединението на взаимодействията става при енергии $E>E_{GUT}$. Най-малката калибровъчна група, която включва групата на СМ и може да обедини силното, слабото и електромагнитното взаимодействие е групата $SU(5)$ \cite{GUT1,GUT2}, като при енергии $E_{weak} << E_{GUT}$ , симетрията на $SU(5)$ е нарушена до симетрията на групата на СМ $SU(3)_C\times SU(2)_L\times U(1)_Y$. Построени са различни варианти на GUT с различни групи на локална симетрия - $SU(5)$, $SO(10)$, $E(6)$ и др. В тези модели се предсказва съществуването на нови заредени и неутрални калибровъчни бозони, както  и на нови частици. В тези модели не се съхраняват ред квантови числа (примерно нарушават се барионното и лептонното квантово число), което води до възможността за наблюдение на нови явления и процеси от типа на разпада на протона. Трябва да отбележим, че досега няма нито едно експериментално потвърждение на предсказанията на GUT.


%Моделът с група $SU(5)$ предсказва и разпадането на протона при енергии $E_{GUT} > 10^{15}$ GeV, което задава и енергетичния мащаб на модела. При енергии от порядъка на $E_{GUT}$ се очаква константите на силното, слабото и електромагнитното взаимодействие да станат почти равни. Равенство между константите на взаимодействие може да бъде постигнато с помощта на някои суперсиметрични сценарии.  Групата $SU(5)$ предрича $24$ бозона, но не предрича допълнителен неутрален калибровъчен бозон. Следваща по-голяма група след $SU(5)$, която включва в себе си групата на СМ и предрича един допълнителен неутрален бозон е групата $SO(10)$. Освен това тази група предполага и съществуването на дясно неутрино. Всички останали теории на великото обединение с групи, по-големи от $SO(10)$, предполагат съществуването на повече от един неутрални калибровъчни бозона, както и съществуването на допълнителни фермиони.
 
\par {\bf Суперсиметрията} \cite{susy1,susy2} е едно от най-популярните разширения на СМ. Суперсиметрията е разширение на групата на Пуанкаре, което води до симетрия между бозони и фермиони и по този начин позволява нетривиално обединение на вътрешните с пространствените симетрии. 
В тези модели на всеки фермион със спин $s$ се съпоставя бозон със спин $s-1/2$ и на всеки бозон със спин $s$ се съпоставя фермион със спин $s-1/2$. По този начин спектърът на частици като минимум се удвоява.

%Съществуват различни суперсиметричи модели, като най-простият от тях е Минималния суперсиметричен модел (МССМ). В МССМ на всеки кирален фермион се съпоставя супесиметричен партньор - сфермион, а на всеки безмасов бозон със спин $\pm1$ се съпоставя суперсиметричен партньор - гейджино със спиралност $\pm 1/2$. 

В суперсиметричните модели, механизмът на нарушаване на електрослабата симетрия изисква въвеждането поне на два хигсови дублета. Нарушението на Суперсиметрията може да бъде извършено по няколко различни механизма, като едни от най-популярните е т.н. супергравитационен модел.
\par Суперсиметрията може да предложи подходящо решение на йерархичния проблем. Предвид на равенството между бозонните и фермионните степени на свобода, квадратичните разходимости при определяне на масата на Хигс бозона могат да бъдат съкратени, тъй като бозонните и фермионните степени на свобода участват с противоположни знаци при пресмятането на радиационните поправки.
\par Обединение на взаимодействията - Поправките, дължащи се на новите полета, въведени в суперсиметричния модел изменят скоростта на изменение на константите на взаимодействие, така, че при определена енергия те да могат да се пресекат в една точка.
\par Обединяването на гравитационното взаимодействие с останалите три е възможно в някои модели с допълнителни измерения. Такъв модел например е предложеният {\bf модел на Калуца и Клайн}, в който гравитационното и електромагнитното взаимодействие могат да бъдат обединени в $5$-мерно пространство. Моделът включва SM плюс допълнителна $U(1)$ симетрия. Допълнителните измерения се компактифицират с достатъчно малък радиус на компактификация, където радиусът е от порядъка на $R \sim 10^{-35}$ m. Моделът предполага безмасов гравитон и безкраен брой възбудени масови състояния на гравитона (Калуца-Клайн възбуждания). При енергии $E<<1/R$ ние сме чувствителни само към безмасовата нулева мода на възбуждане на гравитона. Основен недостатък на този модел е, че той предрича също така и Калуца-Клайн възбуждания на фермионните полета, които не се наблюдават експериментално.
\par Този проблем е отстранен в т.н. {\bf ADD модели} (Arkani-Dimopoulos-Dvali) \cite{ADD}. Аркани, Димопулос и Двали разглеждат идеята, че нашата светът се състои от $D = (4+n)$ измерения, където $n\geqslant 2$ са компактифицирани и време-пространството може да бъде факторизирано:
$R^{4+n}=M^4\times S^n$
В рамките на тези модели полетата от СМ могат да се намират само в $(3+1)$-мерна брана, наречена $3$-брана, а гравитационното поле в цялото време-пространство. По този начин може да се обясни, защо интензитетът на гравитационното взаимодействие е много по-малък в сравнение с другите взаимодействия. 

%в други модели на Рандал и Сундрум които разглеждат идеята, че познатите полета от СМ модел ,,живеят`` само в определено подпространство - брана, а гравитационното поле в цялото време-пространство. Тъй като се намират върху браната, за разлика от гравитона,  полетата от СМ нямат Калуца-Клайн възбуждания.

Моделът предлага и решение на проблема с йерархията на масите, като въвежда така наречената редуцирана маса на Планк:
\begin{equation}
M_{Pl}^2 = M_*^{n+2}V_n
\end{equation}
където $V$ е обемът на пространството, а $M_*$ е редуцираната Планкова маса, т.е. при достатъчно големи размери на допълнителните измерения, масата на Планк може да бъде съществено намалена до мащаби, достижими на съвременните експерименти с високи енергии. Основният проблем на този модел са прекалено големите размери на допълнителните измерения, които са следствие от предположението за компактност на пространството и че то може да бъде факторизирано по измеренията на отделни брани и допълнителните измерения.
\par За разлика от ADD, {\bf моделите на Рандал и Сундрум} (Randall-Sundrun - RS) \cite{RSExrta1, RSExrta2} разглеждат вариант, в който общото време-пространство не може да бъде факторизирано, а напрежението на браните води до закривяване на допълнителните измерения. В този модел редуцирането на масата на Планк се определя основно от кривината $k$ на допълнителните измерения, вместо от техния размер.


\par Необходимо е да се отбележи, че в един кратък обзор не е възможно да бъдат обхванати всички области на физиката на елементарните частици. Тук трябва да се добавят задължително експериментите в областта на неутринната физика а също така и прецизните измервания на параметрите на СМ, физиката на $b$ и $t$ кварките. Към разглежданите теоретични модели могат да се добавят теория на струните, техниколор и други. Голяма част от тези модели предсказват нови частици или явления при енергии от мащаба на TEV. За експерименталното им потвърждаване са необходими ускорители, при които взаимодействията между налитащите частици се случват при много висока енергия и с достатъчно голяма светимост. Най-големият ускорител на частици в света е Големият адронен колайдер (LHC), построен в CERN, който е проектиран с цел да се намери отговор на някои от най-задълбочените въпроси за произхода на Вселената - механизма на получаване на маси от частиците, преобладаването на материята над антиматерията, тъмната материя и тъмната енергия,  кога (при какви енергии) ще се обединят четирите взаимодействия.

\chapter{Големият адронен колайдер LHC}
\input{LHCandCMS/lhcAndCMS.tex}
\chapter{Експериментът CMS}
\input{muonDetector/CMSandMuonSystem.tex}
\chapter{Камери със съпротивителна плоскост}
\input{bobi/RPC_principalOfOperation_1.tex}
\chapter{Тригер на CMS}
\input{bobi/Muon_trigger.tex}
\chapter{Реконструирани мюони}
\input{muonRecoAt7TeV/MuonReco.tex}
\chapter{Работа на системата от RPC на CMS при набор на данни}
\input{cmssw/prompAnalysis.tex}
\chapter{Моделиране на системата от RPC на CMS}
\input{simulation/rpcSim.tex}
\chapter{В търсене на тежки двумюонни резонанси}
\input{diMuReso/diMuResonance.tex}

\chapter{Научни приноси}
\input{prinosi.tex}
\input{myBibl_base.tex}
\chapter{Благодарности}
\input{thanks.tex}
%\chapter{Литература}
\input{myBibl.tex}

\chapter{Приложение А}
\input{simulation/pythia_settingss.tex}
\chapter{Приложение Б}
\input{diMuReso/Appendix_Stat.tex}
%%\chapter{Приложение - цветни фигури}
%%\input{colorFigurees.tex}
%%\chapter{Често използвани съкращения и означения}
%%\input{oznachenia.tex}

\end{document}