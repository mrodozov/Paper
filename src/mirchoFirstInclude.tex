\section{Стандартен модел на елементарните частици}
\par Стандартният модел на елементарните частици (СМ) \cite{PDG,Pich} описва взаимодействието на съставните части на материята. В рамките на СМ, материята е изградена от $12$ фундаментални частици със спин $\frac{1}{2}$ - фермиони, които могат да бъдат обединени в три поколения (виж. фигура \ref{fig:standModel}). Фундаменталните фермиони са лептоните и кварките. Съществуват три лептона с електрически заряд ($-1$) - електрон ($e$), мюон($\mu$) и тау лептон ($\tau)$ и три електрически неутрални лептона - електронно ($\nu_e$), мюонно ($\nu_{\mu}$) и тау ($\nu_{\tau}$) неутрино. Кварките се разделят на два типа - горни и долни. Към горния тип кварки спадат горен ($u$ - up), чаровен ($c$ - charm) и върховен ($t$ - top) кварк, които носят електрически заряд ($+\frac{2}{3}$). Долният тип кварки имат електрически заряд ($-\frac{1}{2}$) и към тях спадат долен ($d$- down), странен ($s$- strange) и красив ($b$ - beauty) кварк.  Всяко едно поколение включва по един долен и един горен кварк и по един електрически зареден и един електрически неутрален лептон. Кварките участват в силните, слабите и електромагнитните взаимодействия. Силното взаимодействие се описва от Квантовата хромодинамика (КХД) \cite{gellmann,Politzer}. КХД е квантово полева теория, която използва неабелева калибровъчна група $SU(3)_C$, където индексът $C$ означава, че преобразованията на групата $SU(3)$ действат върху цветните състояния на кварките. В рамките на КХД кварките имат по три цвята, като взаимодействията между тях са пренасят от осем безмасови калибровъчни бозона - глуони. В следствие на неабелевия характер на калибровъчната група, глуоните също притежават цветен заряд и могат да взаимодействат помежду си. По тази причина константата на силното взаимодействие е по-малка при големи предадени импулси и е по-голяма при малки предадени импулси. Това е причината, поради която кварките не се срещат свободно в природата (т.н. кварков затвор), но образуват безцветни композитни състояния - мезони (състояние кварк-антикварк) и бариони (състояние от три кварка). Мезоните и барионите носят общото название адрони. В типичните протон-протонни взаимодействия при високи енергии, кварките получават достатъчно висок импулс. При опит за ,,раздалечаване`` на кварките от един адрон, енергията на връзката между тях е достатъчна за раждане на нова(нови) двойка(двойки) кварк-антикварк. Характерно за силните взаимодействия е така наречената ,,асимптотична свобода``, което означава, че при много малки разстояния, интензитетът на взаимодействия между кварките и глуоните е много малък, което е причина за възникване на ново състояние на материята - т.н. кварк-глуонна плазма.
\par Заредените лептони участват в електромагнитните и слабите взаимодействия, а неутрината - в слабите. Оказва се, че слабите взаимодействия не могат да бъдат разглеждани независимо от електромагнитните, което налага използването на модел, обединяващ двата типа взаимодействия.